\chapter{ПРАКТИЧЕСКАЯ РАБОТА № 7 <<ДИАГРАММА ДЕЯТЕЛЬНОСТИ>>}

\section{Цель работы}
Целью данной практической работы является ознакомление с семантикой диаграммам деятельности. В ходе работы студенты должны освоить особенности применения диаграмм деятельности для разработки архитектуры ПО.

\section{Теоретические сведения}

Диаграмма деятельности может, при определённых условиях, считаться частным случаем диаграмм состояний, рассмотренных в прошлой практической работе. В отличии от диаграммы состояний, диаграмма деятельности не сосредоточена только на том, чтобы показать для чего система разрабатывается и как она будет использоваться заинтересованными лицами. Основная задача диаграммы деятельности показать какими атомарными действиями будет достигнута та цель, которую формирует каждое заинтересованное лицо. То есть, диаграмма деятельности является, своего рода, поведенческой спецификацией, которая может быть отождествлена с  блок-схемами/сетями Петри в объектно-ориентированном контексте. Диаграммы деятельности позволяют прояснить, как координируется деятельность по предоставлению услуги, как взаимосвязаны события, то есть даёт целостную картину бизнес-процессов предметной области. Таким образом, диаграмма деятельности может быть использована не только для описания всех бизнес-процессов системы, но и для описания требований, предъявляемых к разрабатываемой системе.
На практике наиболее распространено применение диаграмм деятельности для описания на ранних этапах проекта всей бизнес логики, а также для описания системных функций.
Исходя из вышеназванного, можно сделать следующие выводы:
\begin{enumerate}
	\item{ Диаграммы деятельности являются эквивалентом блок-схемам/сетей Петри.}
	\item{ Данный вид поведенческих диаграмм описывает динамический процесс работы системы, путём описания (моделирования) потока управления.}
\end{enumerate}

Стоит учесть, что стандарт UML не описывает всю семантику всех элементов диаграммы деятельности. В зависимости от типа решаемой задачи, используются различные комбинации базовых и расширенных элементов, для достижения требуемого уровня детализации.  

\section{Элементы}

Базовыми элементами диаграммы деятельности являются деятельность, которая используется для представления набора действий и которая представляется на диаграмме, как прямоугольник с закругленными углами, и поток управления, который показывает последовательность выполнения деятельности, который отображается в виде направленной стрелки. Обычно на практике, одна деятельность интерпретирует один шаг алгоритма.

К базовым элементам диаграммы деятельности также относятся начальное состояние, которое отображает начало действий (потоки) и конечное состояние деятельности, которое останавливает весь поток управления данной деятельности.

Также выделяют конечный узел потока, который обозначается как маленький круг с X внутри. Конечный узел потока останавливает только данный поток, не влияя при этом на другие потоки. В некоторых случаях поток может не завершится, а приостановиться на время, в результате некоторого события или выполняться в определённые временные рамки. Примером такой ситуации является чтение данных из файла по таймеру, что показано на рисунке 16. У данного события нет входящих рёбер, поэтому оно включено по умолчанию, так как у него есть одно действие.

В некоторых случаях поток может не просто завершится, а завершиться в результате какого-то события. Обычно ситуацию обрыва потока обозначается в виде молнии.


Также выделяют поток объектов. Поток объектов относится к созданию и модификации объектов по видам деятельности. Стрелка потока объекта от действия к объекту означает, что действие создаёт объект или влияет на него. Стрелка потока объекта от объекта к действию указывает, что действие использует объект.


Расширенными элементами диаграммы активности также являются такие сущности, как отправленные и полученные сигналы, узел принятия решения (ветвление), узел слияния, узел узел соединения, отправленные и полученные сигналы, дорожки (Swimlane).

Сигналы представляются обычно, как действия которые могут быть получены системой извне (из другой системы). Они обычно появляются в парах отправленных и полученных сигналов, потому что состояние не может измениться до тех пор, пока не будет получен ответ на сигнал. Данный механизм похож на синхронные сообщениям на диаграмме последовательности. В качестве примера можно рассмотреть ситуацию, когда моделируется система оплаты для интернет магазина. В данном случае отправляется сигнал запроса на оплату, затем происходит ожидание подтверждения, посредством получения сигнала подтверждения оплаты. Механизм ожидания сигнала подтверждения включается только после того, как будет отправлен сигнал запроса на оплату. Данное расширение было добавлено в версии UML 2.0.

Слияние событий объединяет несколько потоков, которые не являются одновременными.

Узел fork (разделения) потоков используется для разделения поведения на набор параллельных потоков деятельности.

Узел объединения потоков используется для объединения нескольких потоков в один поток деятельности.

Совместное использование улов слияния и разделения часто называют синхронизацией.

«Плавательные дорожки»/Swimlanes представляют собой графические линии, которые разделяют деятельность по различным категориям. Swimlanes группируют связанные действия в одну колонку. В практической деятельности дорожки применяются, как для моделирования бизнес-процессов, так и для описания фукциональных характеристик разрабатываемой системы. Часто дорожки используются для того, чтобы показать роль конкретного пользователя или
В общем виде дорожки могут быть сформированы как по горизонтали, так и по вертикали.

\section{Задание}

Вам, как архитектору программного обеспечения, поступила задача на разработку архитектуры информационной системы для автоматизации библиотечной деятельности, которая позволит вести централизированный учёт, выданных книг, регистрационных карточек посетителей, а также контролировать новые поступления в библиотеку. Необходимо разработать диаграмму деятельности по предоставлению услуги выдачи книги новому пользователю библиотеки. Для этого необходимо изучить предметную область, а также все бизнес процессы библиотеки. По желанию, можно расширить функционал системы, например, добавив возможность вносить в библиотечный фонд документы поступающие по почте для их хранения в библиотечных фондах, и нарисовать соответствующие диаграммы деятельности.

\section{Выполнение практического задания}

\begin{figure}[h!]
        \centering
        \includegraphics[width=0.7\textwidth]{images/7/7.eps}
        \caption{Диаграмма Активности}
\end{figure}

\section{Вывод}

Ознакомились с семантикой диаграммам деятельности, освоили особенности применения диаграмм деятельности для разработки архитектуры ПО.
