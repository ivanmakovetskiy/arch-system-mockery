\chapter {ПРАКТИЧЕСКАЯ РАБОТА № 1 <<ВЫЯВЛЕНИЕ ЗАИНТЕРЕСОВАННЫХ ЛИЦ В ИТ-ПРОЕКТАХ>>}

\section{Цель работы}

Сформировать навыки работы с реальными заказчиками программных систем; идентификации заинтересованных лиц и интервью с ними; анализа полученного материала; формулирования проблемы, ее актуальности и потребностей заинтересованных лиц.

\section{Краткие теоретические сведения}

На этапе анализа проблемы проводится анализ предметной области, для которой разрабатывается ПО.

Цели этапа:

\begin{enumerate}
	\item{Определение границ или контура системы.}
	\item{Описание объектов автоматизации и / или формализации данных об этих объектах.}
	\item{Выявление или определение потребностей заказчика ПО.}
\end{enumerate}

Анализ предметной области можно проводить, например, основываясь на теории системного анализа и использовать предложенные в ней методы.

Исходными данными для этапа системного анализа являются:

\begin{enumerate}
	\item{Регламенты работы отделов и должностные инструкции сотрудников этих отделов.}
	\item{Анкеты опроса заинтересованных лиц.}
	\item{Записи интервью с заинтересованными лицами.}
	\item{Другие документы, имеющие отношение к исследуемому объекту.}
\end{enumerate}

Выходными данными, или результатом, этапа системного анализа являются:

\begin{enumerate}
	\item{Перечень заинтересованных лиц.}
	\item{Список потребностей заинтересованных лиц в разрабатываемом ПО.}
	\item{Описание объектов автоматизации.}
	\item{Модель объектов автоматизации или предметной области.}
\end{enumerate}

\section{Задание}

\begin{enumerate}
	\item{Составить перечень заинтересованных лиц.}
	\item{Провести интервью и/или анкетирование с каждым заинтересованным лицом.}
	\item{Проанализировать полученную информацию и сформулировать актуальность проблемы и потребности заинтересованных лиц.}
\end{enumerate}

\section{Выполнение практической работы}

В качестве объекта автоматизации выбран ресторан армянской кухни имени Мартина Хайдеггера «Этр лю’ля».

\subsection{Перечень заинтересованных лиц}

\begin{enumerate}
	\item{Главный кондитер.}
	\item{Менеджер по персоналу.}
\end{enumerate}

\subsection{Интервью с заинтересованными лицами}

\subsubsection{Интервью с главным кондитером}

\begin{enumerate}
	\item{Имя. Хуциевян, Марлен Мартынович.}
	\item{Наименование организации. ООО «Ресторан армянской кухни имени Мартина Хайдеггера l’Être­là.»}
	\item{Наименование структурного подразделения. Кухня, кондитерский цех.}
	\item{Должность. Главный кондитер.}
	\item{Кому Вы непосредственно подчиняетесь? Исполняющему директору ресторана.}
	\item{Каковы Ваши основные обязанности? Контроль десертов на соответствие внутренним регламентам ресторана, административное управление подчиненными, производство десертов.}
	\item{Что Вы в основном производите? Гата, нузук, суджук, пахлаву по-еревански, торт «Микадо».}
	\item{Для кого? Посетителей, проверяющих и выручки директора ресторана.}
	\item{Какие документы или какую информацию можно считать входящими, или необходимыми, для Вашей деятельности? Рецепты и входящее сообщение от официанта с заказом посетителя, а также распоряжения руководства.}
	\item{Какие документы или какую информацию можно считать исходящими, или результатом Вашей деятельности? Сообщение официанту о статусе изготовления продукта, отчет о необходимых ингредиентах управляющему.}
	\item{Как измеряется успех Вашей деятельности? Качественным образом --- в отзывах клиентов. Количественных метрик для описания вкуса торта «Микадо» найти не удалось.}
	\item{Какие проблемы влияют на успешность Вашей деятельности? Несвежие ингредиенты, плохое положение республики Армения на политической арене, неопытные кондитеры в моем подчинении, etc, etc.}
	\item{ Какие тенденции, если такие существуют, делают Вашу работу проще или сложнее? Всеобщая механизация ручного труда и развитие дигитального ресторанного бизнеса с одной стороны и повальное ухудшение качества образования кондитеров с другой.}
	\item{ Какой интерес или какие потребности у Вас есть относительно бу­ дущего решения (разрабатываемого ПО)? Я надеюсь, что при разработке ПО найдутся количественные метрики для описания вкуса торта «Микадо».}
\end{enumerate}

\subsubsection{Интервью с менеджером по персоналу}

\begin{enumerate}
	\item{Имя. Смотрящих Артем Артемович.}
	\item{Наименование организации. ООО «Ресторан армянской кухни имени Мартина Хайдеггера l’Être là».}
	\item{Наименование структурного подразделения. Административный персонал.}
	\item{Должность. Менеджер по персоналу.}
	\item{Кому Вы непосредственно подчиняетесь? Исполняющему директору ресторана.}
	\item{Каковы Ваши основные обязанности? Распределение обязанностей между сотрудниками, обучение персонала, подготовка заведения к открытию и закрытию, приём гостей --- если в заведении нет хостес.}
	\item{Что Вы в основном производите? Контроль и регуляция правильной работы и функционирования всех работников кухни и зала.}
	\item{Для кого? Посетителей, контролирующих органов и директора организации.}
	\item{Какие документы или какую информацию можно считать входящими, или необходимыми, для Вашей деятельности? Зарезервированные столики, время резерва, ФИО посетителя, состав персонала, продукты в наличии.}
	\item{ Какие документы или какую информацию можно считать исходящими, или результатом Вашей деятельности? Прошедшие посетители, сумма заказа, официант, обслуживший гостя, результат и возможный отзыв.}
	\item{ Как измеряется успех Вашей деятельности? Лучшее отражение результата работы --- отзывы клиентов и отсутствие потерянной выручки.}
	\item{ Какие проблемы влияют на успешность Вашей деятельности? Невыход персонала на работу, посетитель не явился/опоздал на бронированный столик.}
	\item{ Какие тенденции, если такие существуют, делают Вашу работу проще или сложнее? Соблюдение графика со стороны как работников, так и посетителей, отсутствие форс­мажоров.}
	\item{ Какой интерес или какие потребности у Вас есть относительно будущего решения (разрабатываемого ПО)? Быстрые, насколько возможно, уведомления о заказах и бронировании, а также обновления статуса занятых столов и блюд, которые есть в наличии.}
\end{enumerate}

\section{Анализ полученной информации}

В результате анкетирования и интервьюирования всех заинтересованных
лиц были сформулированы потребности заказчика относительно разрабатыва­
емого ПО.
Среди них основными можно назвать:

\begin{enumerate}
	\item{Реализовать возможность оценивания готовых блюд.}
	\item{Реализовать возможность оценивания обслуживания и отзывов.}
	\item{Автоматизировать систему учёта свободных столов и работу уведомлений о бронировании.­}
	\item{Наладить меню работы ресторана на работу в реальном времени.}
\end{enumerate}

\section{Вывод}

В результате выполнения работы были сформированы навыки работы с реальными заказчиками программных систем; идентификации заинтересованных лиц и интервью с ними; анализа полученного материала; формулирования проблемы, ее актуальности и потребностей заинтересованных лиц.
